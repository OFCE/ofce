% Options for packages loaded elsewhere
\PassOptionsToPackage{unicode}{hyperref}
\PassOptionsToPackage{hyphens}{url}
\PassOptionsToPackage{dvipsnames,svgnames,x11names}{xcolor}
%
\documentclass[
  11pt,
  a4paper,
  DIV=10]{scrreprt}

\usepackage{amsmath,amssymb}
\usepackage{iftex}
\ifPDFTeX
  \usepackage[T1]{fontenc}
  \usepackage[utf8]{inputenc}
  \usepackage{textcomp} % provide euro and other symbols
\else % if luatex or xetex
  \usepackage{unicode-math}
  \defaultfontfeatures{Scale=MatchLowercase}
  \defaultfontfeatures[\rmfamily]{Ligatures=TeX,Scale=1}
\fi
\usepackage{lmodern}
\ifPDFTeX\else  
    % xetex/luatex font selection
\fi
% Use upquote if available, for straight quotes in verbatim environments
\IfFileExists{upquote.sty}{\usepackage{upquote}}{}
\IfFileExists{microtype.sty}{% use microtype if available
  \usepackage[]{microtype}
  \UseMicrotypeSet[protrusion]{basicmath} % disable protrusion for tt fonts
}{}
\makeatletter
\@ifundefined{KOMAClassName}{% if non-KOMA class
  \IfFileExists{parskip.sty}{%
    \usepackage{parskip}
  }{% else
    \setlength{\parindent}{0pt}
    \setlength{\parskip}{6pt plus 2pt minus 1pt}}
}{% if KOMA class
  \KOMAoptions{parskip=half}}
\makeatother
\usepackage{xcolor}
\setlength{\emergencystretch}{3em} % prevent overfull lines
\setcounter{secnumdepth}{3}
% Make \paragraph and \subparagraph free-standing
\ifx\paragraph\undefined\else
  \let\oldparagraph\paragraph
  \renewcommand{\paragraph}[1]{\oldparagraph{#1}\mbox{}}
\fi
\ifx\subparagraph\undefined\else
  \let\oldsubparagraph\subparagraph
  \renewcommand{\subparagraph}[1]{\oldsubparagraph{#1}\mbox{}}
\fi


\providecommand{\tightlist}{%
  \setlength{\itemsep}{0pt}\setlength{\parskip}{0pt}}\usepackage{longtable,booktabs,array}
\usepackage{calc} % for calculating minipage widths
% Correct order of tables after \paragraph or \subparagraph
\usepackage{etoolbox}
\makeatletter
\patchcmd\longtable{\par}{\if@noskipsec\mbox{}\fi\par}{}{}
\makeatother
% Allow footnotes in longtable head/foot
\IfFileExists{footnotehyper.sty}{\usepackage{footnotehyper}}{\usepackage{footnote}}
\makesavenoteenv{longtable}
\usepackage{graphicx}
\makeatletter
\def\maxwidth{\ifdim\Gin@nat@width>\linewidth\linewidth\else\Gin@nat@width\fi}
\def\maxheight{\ifdim\Gin@nat@height>\textheight\textheight\else\Gin@nat@height\fi}
\makeatother
% Scale images if necessary, so that they will not overflow the page
% margins by default, and it is still possible to overwrite the defaults
% using explicit options in \includegraphics[width, height, ...]{}
\setkeys{Gin}{width=\maxwidth,height=\maxheight,keepaspectratio}
% Set default figure placement to htbp
\makeatletter
\def\fps@figure{htbp}
\makeatother

\usepackage{booktabs}
\usepackage{caption}
\usepackage{longtable}
\usepackage{colortbl}
\usepackage{array}
\makeatletter
\@ifpackageloaded{tcolorbox}{}{\usepackage[skins,breakable]{tcolorbox}}
\@ifpackageloaded{fontawesome5}{}{\usepackage{fontawesome5}}
\definecolor{quarto-callout-color}{HTML}{909090}
\definecolor{quarto-callout-note-color}{HTML}{0758E5}
\definecolor{quarto-callout-important-color}{HTML}{CC1914}
\definecolor{quarto-callout-warning-color}{HTML}{EB9113}
\definecolor{quarto-callout-tip-color}{HTML}{00A047}
\definecolor{quarto-callout-caution-color}{HTML}{FC5300}
\definecolor{quarto-callout-color-frame}{HTML}{acacac}
\definecolor{quarto-callout-note-color-frame}{HTML}{4582ec}
\definecolor{quarto-callout-important-color-frame}{HTML}{d9534f}
\definecolor{quarto-callout-warning-color-frame}{HTML}{f0ad4e}
\definecolor{quarto-callout-tip-color-frame}{HTML}{02b875}
\definecolor{quarto-callout-caution-color-frame}{HTML}{fd7e14}
\makeatother
\makeatletter
\@ifpackageloaded{caption}{}{\usepackage{caption}}
\AtBeginDocument{%
\ifdefined\contentsname
  \renewcommand*\contentsname{Table des matières}
\else
  \newcommand\contentsname{Table des matières}
\fi
\ifdefined\listfigurename
  \renewcommand*\listfigurename{Figures}
\else
  \newcommand\listfigurename{Figures}
\fi
\ifdefined\listtablename
  \renewcommand*\listtablename{Tableaux}
\else
  \newcommand\listtablename{Tableaux}
\fi
\ifdefined\figurename
  \renewcommand*\figurename{Figure}
\else
  \newcommand\figurename{Figure}
\fi
\ifdefined\tablename
  \renewcommand*\tablename{Tableau}
\else
  \newcommand\tablename{Tableau}
\fi
}
\@ifpackageloaded{float}{}{\usepackage{float}}
\floatstyle{ruled}
\@ifundefined{c@chapter}{\newfloat{codelisting}{h}{lop}}{\newfloat{codelisting}{h}{lop}[chapter]}
\floatname{codelisting}{Listing}
\newcommand*\listoflistings{\listof{codelisting}{Liste des Listings}}
\makeatother
\makeatletter
\makeatother
\makeatletter
\@ifpackageloaded{caption}{}{\usepackage{caption}}
\@ifpackageloaded{subcaption}{}{\usepackage{subcaption}}
\makeatother
\ifLuaTeX
\usepackage[bidi=basic]{babel}
\else
\usepackage[bidi=default]{babel}
\fi
\babelprovide[main,import]{french}
% get rid of language-specific shorthands (see #6817):
\let\LanguageShortHands\languageshorthands
\def\languageshorthands#1{}
\ifLuaTeX
  \usepackage{selnolig}  % disable illegal ligatures
\fi
\IfFileExists{bookmark.sty}{\usepackage{bookmark}}{\usepackage{hyperref}}
\IfFileExists{xurl.sty}{\usepackage{xurl}}{} % add URL line breaks if available
\urlstyle{same} % disable monospaced font for URLs
\hypersetup{
  pdftitle={Table in a div},
  pdflang={fr},
  colorlinks=true,
  linkcolor={blue},
  filecolor={Maroon},
  citecolor={Blue},
  urlcolor={Blue},
  pdfcreator={LaTeX via pandoc}}

%%% title.tex we use to call other packages. Because this works
\usepackage{titlepic}
\usepackage{titling}
\usepackage{graphicx}
\usepackage{fancyhdr}
\usepackage{fontspec}
\usepackage{placeins}
\usepackage{graphbox}
\usepackage{tikz}
\usepackage{geometry}
\usepackage{xcolor}
\usepackage{amsmath}
\usepackage[some]{background}
\usepackage{lipsum}
\usepackage{caption}
\usepackage{datetime}
\usepackage{xstring}

\setmainfont{OpenSans}[
  UprightFont = {*-Regular},
  BoldFont = {*-Bold},
  BoldItalicFont = {*-BoldItalic},
  ItalicFont = {*-Italic},
  Path = {_extensions/ofce/wp/OpenSans/},
  Extension = {.ttf}
]
\setsansfont{OpenSans}[
  UprightFont = {*-Regular},
  BoldFont = {*-Bold},
  BoldItalicFont = {*-BoldItalic},
  ItalicFont = {*-Italic},
  Path = {_extensions/ofce/wp/OpenSans/},
  Extension = {.ttf}
]
\def\getYear#1{\StrLeft{#1}{4}}
\def\getMonth#1{\StrMid{#1}{6}{7}}
\def\getDay#1{\StrRight{#1}{2}}

\def\jolimois#1{\monthname{\getMonth{#1}}}
\begin{document}



\definecolor{ofcepbbleu}{RGB}{1, 97, 131}
\definecolor{ofcerouge}{RGB}{198, 45, 43}

\begin{titlepage}
  \backgroundsetup{
    scale=1,
    angle=0,
    opacity=1,
    contents={
  \begin{tikzpicture}[remember picture,overlay]
    \useasboundingbox (0,0) rectangle(\the\paperwidth,\the\paperheight);
      \node [anchor = west] at (-8.25,12.5) {\includegraphics[width=2.5cm]{\_extensions/ofce/wp/ofce\_m.png}};
      \node [anchor = west] at (-8.5,-12.75){\includegraphics[width=3cm]{\_extensions/ofce/wp/sciencespo.png}};
      \node [anchor = east] at (8.5,-11.5){\textcolor{black}{Date de première publication : }};
      \node [anchor = east] at (8.5,-12){\textcolor{black}{Date de dernière modification : }};
      \node [anchor = east] at (8.2,13-0.7){\textcolor{gray}{\Huge\textit{Working Paper}}};
      \draw [thick,black](-5.25,-13) -- (-5.25,13);
      \draw [color = white, fill=ofcepbbleu] (8.7,11.4) rectangle (10.45,13.15);
      \draw [color = white, fill=ofcepbbleu, very thick] (8.75-.5,11.25-.5) rectangle (8.75+.6,11.25+.6);
      \node [anchor = east] at (11-0.7,13-0.7){\textcolor{white}{\huge\textbf{}}};
      \node [anchor = east] at (10-0.74,12-0.67){\textcolor{white}{\textbf{2023}}};
    \end{tikzpicture}
  }
}
\BgThispage

\hspace{3cm}
\begin{minipage}{12.5cm}
  \vspace{5cm}
  \textcolor{ofcerouge}{\Huge\textbf{\textsf{Table in a div}}}

    \vspace{20mm}

  %
   % by-author
 \end{minipage}


%
%
\newpage
%% deuxieme page
\pagestyle{empty}
\begin{minipage}{\linewidth}

  \includegraphics[width=2cm]{\_extensions/ofce/wp/ofce\_m.png}

  \includegraphics[width=2cm]{\_extensions/ofce/wp/sciencespo.png}

\end{minipage}

\vspace{2cm}

\LARGE\textbf{Table in a div}


\large\textbf{}

\end{titlepage}\renewcommand*\contentsname{Table des matières}
{
\hypersetup{linkcolor=}
\setcounter{tocdepth}{0}
\tableofcontents
}
\begin{table}

\caption{\label{tbl-table1}}

\centering{

\captionsetup{labelsep=none}

\begin{longtable*}{lrl}
\toprule
a & b & c \\ 
\midrule\addlinespace[2.5pt]
text & 1 & Long text aaaaaaaaaaaaaaaaaaaaaaaaaaaaaaaaaaaaargh \\ 
others & 2 & Long text aaaaaaaaaaaaaaaaaaaaaaaaaaaaaaaaaaaaargh \\ 
\bottomrule
\end{longtable*}

}

\end{table}%

texte avant

\begin{tcolorbox}[enhanced jigsaw, titlerule=0mm, toprule=.15mm, colframe=quarto-callout-tip-color-frame, toptitle=1mm, bottomrule=.15mm, rightrule=.15mm, coltitle=black, bottomtitle=1mm, leftrule=.75mm, opacityback=0, title={titre de l'encadré}, arc=.35mm, opacitybacktitle=0.6, left=2mm, colbacktitle=quarto-callout-tip-color!10!white, colback=white, breakable]

blablbl

\phantomsection\label{tbl-table2}
\begin{longtable}[]{@{}lrl@{}}
\toprule\noalign{}
a & b & c \\
\midrule\noalign{}
\endhead
\bottomrule\noalign{}
\endlastfoot
text & 1 & Long text aaaaaaaaaaaaaaaaaaaaaaaaaaaaaaaaaaaaargh \\
others & 2 & Long text aaaaaaaaaaaaaaaaaaaaaaaaaaaaaaaaaaaaargh \\
\end{longtable}

\end{tcolorbox}



\end{document}
