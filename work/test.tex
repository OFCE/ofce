% Options for packages loaded elsewhere
\PassOptionsToPackage{unicode}{hyperref}
\PassOptionsToPackage{hyphens}{url}
\PassOptionsToPackage{dvipsnames,svgnames,x11names}{xcolor}
%
\documentclass[
  8pt,
  a4paper,
  DIV=10]{scrreprt}

\usepackage{amsmath,amssymb}
\usepackage{iftex}
\ifPDFTeX
  \usepackage[T1]{fontenc}
  \usepackage[utf8]{inputenc}
  \usepackage{textcomp} % provide euro and other symbols
\else % if luatex or xetex
  \usepackage{unicode-math}
  \defaultfontfeatures{Scale=MatchLowercase}
  \defaultfontfeatures[\rmfamily]{Ligatures=TeX,Scale=1}
\fi
\usepackage{lmodern}
\ifPDFTeX\else  
    % xetex/luatex font selection
\fi
% Use upquote if available, for straight quotes in verbatim environments
\IfFileExists{upquote.sty}{\usepackage{upquote}}{}
\IfFileExists{microtype.sty}{% use microtype if available
  \usepackage[]{microtype}
  \UseMicrotypeSet[protrusion]{basicmath} % disable protrusion for tt fonts
}{}
\makeatletter
\@ifundefined{KOMAClassName}{% if non-KOMA class
  \IfFileExists{parskip.sty}{%
    \usepackage{parskip}
  }{% else
    \setlength{\parindent}{0pt}
    \setlength{\parskip}{6pt plus 2pt minus 1pt}}
}{% if KOMA class
  \KOMAoptions{parskip=half}}
\makeatother
\usepackage{xcolor}
\setlength{\emergencystretch}{3em} % prevent overfull lines
\setcounter{secnumdepth}{3}
% Make \paragraph and \subparagraph free-standing
\ifx\paragraph\undefined\else
  \let\oldparagraph\paragraph
  \renewcommand{\paragraph}[1]{\oldparagraph{#1}\mbox{}}
\fi
\ifx\subparagraph\undefined\else
  \let\oldsubparagraph\subparagraph
  \renewcommand{\subparagraph}[1]{\oldsubparagraph{#1}\mbox{}}
\fi

\usepackage{color}
\usepackage{fancyvrb}
\newcommand{\VerbBar}{|}
\newcommand{\VERB}{\Verb[commandchars=\\\{\}]}
\DefineVerbatimEnvironment{Highlighting}{Verbatim}{commandchars=\\\{\}}
% Add ',fontsize=\small' for more characters per line
\usepackage{framed}
\definecolor{shadecolor}{RGB}{241,243,245}
\newenvironment{Shaded}{\begin{snugshade}}{\end{snugshade}}
\newcommand{\AlertTok}[1]{\textcolor[rgb]{0.68,0.00,0.00}{#1}}
\newcommand{\AnnotationTok}[1]{\textcolor[rgb]{0.37,0.37,0.37}{#1}}
\newcommand{\AttributeTok}[1]{\textcolor[rgb]{0.40,0.45,0.13}{#1}}
\newcommand{\BaseNTok}[1]{\textcolor[rgb]{0.68,0.00,0.00}{#1}}
\newcommand{\BuiltInTok}[1]{\textcolor[rgb]{0.00,0.23,0.31}{#1}}
\newcommand{\CharTok}[1]{\textcolor[rgb]{0.13,0.47,0.30}{#1}}
\newcommand{\CommentTok}[1]{\textcolor[rgb]{0.37,0.37,0.37}{#1}}
\newcommand{\CommentVarTok}[1]{\textcolor[rgb]{0.37,0.37,0.37}{\textit{#1}}}
\newcommand{\ConstantTok}[1]{\textcolor[rgb]{0.56,0.35,0.01}{#1}}
\newcommand{\ControlFlowTok}[1]{\textcolor[rgb]{0.00,0.23,0.31}{#1}}
\newcommand{\DataTypeTok}[1]{\textcolor[rgb]{0.68,0.00,0.00}{#1}}
\newcommand{\DecValTok}[1]{\textcolor[rgb]{0.68,0.00,0.00}{#1}}
\newcommand{\DocumentationTok}[1]{\textcolor[rgb]{0.37,0.37,0.37}{\textit{#1}}}
\newcommand{\ErrorTok}[1]{\textcolor[rgb]{0.68,0.00,0.00}{#1}}
\newcommand{\ExtensionTok}[1]{\textcolor[rgb]{0.00,0.23,0.31}{#1}}
\newcommand{\FloatTok}[1]{\textcolor[rgb]{0.68,0.00,0.00}{#1}}
\newcommand{\FunctionTok}[1]{\textcolor[rgb]{0.28,0.35,0.67}{#1}}
\newcommand{\ImportTok}[1]{\textcolor[rgb]{0.00,0.46,0.62}{#1}}
\newcommand{\InformationTok}[1]{\textcolor[rgb]{0.37,0.37,0.37}{#1}}
\newcommand{\KeywordTok}[1]{\textcolor[rgb]{0.00,0.23,0.31}{#1}}
\newcommand{\NormalTok}[1]{\textcolor[rgb]{0.00,0.23,0.31}{#1}}
\newcommand{\OperatorTok}[1]{\textcolor[rgb]{0.37,0.37,0.37}{#1}}
\newcommand{\OtherTok}[1]{\textcolor[rgb]{0.00,0.23,0.31}{#1}}
\newcommand{\PreprocessorTok}[1]{\textcolor[rgb]{0.68,0.00,0.00}{#1}}
\newcommand{\RegionMarkerTok}[1]{\textcolor[rgb]{0.00,0.23,0.31}{#1}}
\newcommand{\SpecialCharTok}[1]{\textcolor[rgb]{0.37,0.37,0.37}{#1}}
\newcommand{\SpecialStringTok}[1]{\textcolor[rgb]{0.13,0.47,0.30}{#1}}
\newcommand{\StringTok}[1]{\textcolor[rgb]{0.13,0.47,0.30}{#1}}
\newcommand{\VariableTok}[1]{\textcolor[rgb]{0.07,0.07,0.07}{#1}}
\newcommand{\VerbatimStringTok}[1]{\textcolor[rgb]{0.13,0.47,0.30}{#1}}
\newcommand{\WarningTok}[1]{\textcolor[rgb]{0.37,0.37,0.37}{\textit{#1}}}

\providecommand{\tightlist}{%
  \setlength{\itemsep}{0pt}\setlength{\parskip}{0pt}}\usepackage{longtable,booktabs,array}
\usepackage{calc} % for calculating minipage widths
% Correct order of tables after \paragraph or \subparagraph
\usepackage{etoolbox}
\makeatletter
\patchcmd\longtable{\par}{\if@noskipsec\mbox{}\fi\par}{}{}
\makeatother
% Allow footnotes in longtable head/foot
\IfFileExists{footnotehyper.sty}{\usepackage{footnotehyper}}{\usepackage{footnote}}
\makesavenoteenv{longtable}
\usepackage{graphicx}
\makeatletter
\def\maxwidth{\ifdim\Gin@nat@width>\linewidth\linewidth\else\Gin@nat@width\fi}
\def\maxheight{\ifdim\Gin@nat@height>\textheight\textheight\else\Gin@nat@height\fi}
\makeatother
% Scale images if necessary, so that they will not overflow the page
% margins by default, and it is still possible to overwrite the defaults
% using explicit options in \includegraphics[width, height, ...]{}
\setkeys{Gin}{width=\maxwidth,height=\maxheight,keepaspectratio}
% Set default figure placement to htbp
\makeatletter
\def\fps@figure{htbp}
\makeatother

\usepackage{booktabs}
\usepackage{caption}
\usepackage{longtable}
\usepackage{colortbl}
\usepackage{array}
\makeatletter
\@ifpackageloaded{tcolorbox}{}{\usepackage[skins,breakable]{tcolorbox}}
\@ifpackageloaded{fontawesome5}{}{\usepackage{fontawesome5}}
\definecolor{quarto-callout-color}{HTML}{909090}
\definecolor{quarto-callout-note-color}{HTML}{0758E5}
\definecolor{quarto-callout-important-color}{HTML}{CC1914}
\definecolor{quarto-callout-warning-color}{HTML}{EB9113}
\definecolor{quarto-callout-tip-color}{HTML}{00A047}
\definecolor{quarto-callout-caution-color}{HTML}{FC5300}
\definecolor{quarto-callout-color-frame}{HTML}{acacac}
\definecolor{quarto-callout-note-color-frame}{HTML}{4582ec}
\definecolor{quarto-callout-important-color-frame}{HTML}{d9534f}
\definecolor{quarto-callout-warning-color-frame}{HTML}{f0ad4e}
\definecolor{quarto-callout-tip-color-frame}{HTML}{02b875}
\definecolor{quarto-callout-caution-color-frame}{HTML}{fd7e14}
\makeatother
\makeatletter
\@ifpackageloaded{caption}{}{\usepackage{caption}}
\AtBeginDocument{%
\ifdefined\contentsname
  \renewcommand*\contentsname{Table des matières}
\else
  \newcommand\contentsname{Table des matières}
\fi
\ifdefined\listfigurename
  \renewcommand*\listfigurename{Figures}
\else
  \newcommand\listfigurename{Figures}
\fi
\ifdefined\listtablename
  \renewcommand*\listtablename{Tableaux}
\else
  \newcommand\listtablename{Tableaux}
\fi
\ifdefined\figurename
  \renewcommand*\figurename{Figure}
\else
  \newcommand\figurename{Figure}
\fi
\ifdefined\tablename
  \renewcommand*\tablename{Tableau}
\else
  \newcommand\tablename{Tableau}
\fi
}
\@ifpackageloaded{float}{}{\usepackage{float}}
\floatstyle{ruled}
\@ifundefined{c@chapter}{\newfloat{codelisting}{h}{lop}}{\newfloat{codelisting}{h}{lop}[chapter]}
\floatname{codelisting}{Listing}
\newcommand*\listoflistings{\listof{codelisting}{Liste des Listings}}
\makeatother
\makeatletter
\makeatother
\makeatletter
\@ifpackageloaded{caption}{}{\usepackage{caption}}
\@ifpackageloaded{subcaption}{}{\usepackage{subcaption}}
\makeatother
\ifLuaTeX
\usepackage[bidi=basic]{babel}
\else
\usepackage[bidi=default]{babel}
\fi
\babelprovide[main,import]{french}
% get rid of language-specific shorthands (see #6817):
\let\LanguageShortHands\languageshorthands
\def\languageshorthands#1{}
\ifLuaTeX
  \usepackage{selnolig}  % disable illegal ligatures
\fi
\IfFileExists{bookmark.sty}{\usepackage{bookmark}}{\usepackage{hyperref}}
\IfFileExists{xurl.sty}{\usepackage{xurl}}{} % add URL line breaks if available
\urlstyle{same} % disable monospaced font for URLs
\hypersetup{
  pdftitle={Table in a div},
  pdflang={fr},
  colorlinks=true,
  linkcolor={blue},
  filecolor={Maroon},
  citecolor={Blue},
  urlcolor={Blue},
  pdfcreator={LaTeX via pandoc}}

%%% title.tex we use to call other packages. Because this works
\usepackage{titlepic}
\usepackage{titling}
\usepackage{graphicx}
\usepackage{fontspec}
\usepackage{placeins}
\usepackage{graphbox}
\usepackage{tikz}
\usepackage{geometry}
\usepackage{xcolor}
\usepackage{amsmath}
\usepackage[some]{background}
\usepackage{lipsum}
\usepackage{caption}
\usepackage{datetime}
\usepackage{xstring}
\usepackage{blindtext}
\usepackage{scrlayer-scrpage}

\setmainfont{OpenSans}[
  UprightFont = {*-Regular},
  BoldFont = {*-Bold},
  BoldItalicFont = {*-BoldItalic},
  ItalicFont = {*-Italic},
  Path = {_extensions/ofce/wp/OpenSans/},
  Extension = {.ttf}
]
\setsansfont{OpenSans}[
  UprightFont = {*-Regular},
  BoldFont = {*-Bold},
  BoldItalicFont = {*-BoldItalic},
  ItalicFont = {*-Italic},
  Path = {_extensions/ofce/wp/OpenSans/},
  Extension = {.ttf}
]

%% \KOMAoption{fontsize 8pt}

\def\getYear#1{\StrLeft{#1}{4}}
\def\getMonth#1{\StrMid{#1}{6}{7}}
\def\getDay#1{\StrRight{#1}{2}}

\def\jolimois#1{\monthname{\getMonth{#1}}}

\definecolor{quarto-callout-color}{HTML}{eeeeee}
\definecolor{quarto-callout-tip-color}{HTML}{dddddd}
\definecolor{quarto-callout-tip-color-frame}{HTML}{eeeeee}

\clearpairofpagestyles

\KOMAoptions{headsepline=true, twoside=true}

\pagestyle{scrheadings}
\setkomafont{pageheadfoot}{\small}

\rehead{Document de travail n°1xx}
\lohead{\includegraphics[height=0.25cm]{\_extensions/ofce/wp/ofce\_m.png}}

\lofoot*{\thepage}
\refoot*{\thepage}
\begin{document}



\definecolor{ofcepbbleu}{RGB}{1, 97, 131}
\definecolor{ofcerouge}{RGB}{198, 45, 43}
\definecolor{scporouge}{RGB}{231, 0, 26}

\begin{titlepage}
  \backgroundsetup{
    scale=1,
    angle=0,
    opacity=1,
    contents={
  \begin{tikzpicture}[remember picture,overlay]
    \useasboundingbox (0,0) rectangle(\the\paperwidth,\the\paperheight);
      \node [anchor = center] at (-2.875-5.25,12.5) {\includegraphics[width=2.5cm]{\_extensions/ofce/wp/ofce\_m.png}};
      \node [anchor = center] at (-2.875-5.25,-12.75){\includegraphics[width=2.5cm]{\_extensions/ofce/wp/sciencespo.png}};
      \node [anchor = east] at (8.5,-11.5){\textcolor{black}{Date de première publication : }};
      \node [anchor = east] at (8.5,-12){\textcolor{black}{Date de dernière modification : }};
      \node [anchor = east] at (8.5,13-0.7){\textcolor{gray}{\Huge\textit{Working Paper}}};
      \draw [thick,black](-5.75,-13) -- (-5.75,13);
      \draw [color = white, fill=ofcepbbleu] (8.7,11.4) rectangle (10.45,13.15);
      \draw [color = white, fill=ofcepbbleu, very thick] (8.75-.5,11.25-.5) rectangle (8.75+.6,11.25+.6);
      \node [anchor = east] at (11-0.7,13-0.7){\textcolor{white}{\huge\textbf{1xx}}};
      \node [anchor = east] at (10-0.74,12-0.67){\textcolor{white}{\textbf{2023}}};
    \end{tikzpicture}
  }
}
\BgThispage

\hspace{4cm}
\begin{minipage}{12.5cm}
  \vspace{5cm}
  \begin{flushleft}
  \textcolor{scporouge}{\Huge\textbf{\textsf{Table in a div}}}
  
   \vspace{5mm}
  \textcolor{scporouge}{\large\textbf{\textsf{hello div}}}
 \vspace{5mm}
    \vspace{20mm}
  \end{flushleft}
  
   % by-author
 \end{minipage}


%
%
\newpage
%% deuxieme page
\pagestyle{empty}
%%  \raisebox{3cm}{\begin{minipage}{\linewidth}
%%  \includegraphics[width=2cm]{\_extensions/ofce/wp/ofce\_m.png}
%%
%%  \includegraphics[width=2cm]{\_extensions/ofce/wp/sciencespo.png}
%%  \end{minipage}}
%%
 
\LARGE\textbf{Table in a div}

\large\textbf{hello div}

\vspace{1cm}


\par\rule{\textwidth}{0.5pt}

Lorem ipsum dolor sit amet, consectetur adipiscing elit. Sed ac augue
quis velit varius cursus. Pellentesque a auctor lectus, at auctor neque.
Maecenas lorem mauris, varius nec metus vel, varius pulvinar augue.
Phasellus eget nisl commodo nunc vestibulum venenatis sed id lorem. Ut
rutrum pretium dolor quis sagittis. Proin porttitor mollis fringilla.
Praesent sed convallis odio. Vestibulum consequat, libero auctor
hendrerit blandit, purus erat laoreet dolor, sit amet pellentesque neque
arcu nec elit. Curabitur feugiat, dolor efficitur dapibus fermentum,
eros lacus gravida dui, eget mattis magna orci quis dui. Etiam viverra
ante lacus, quis congue lorem euismod sed. Proin sed elit interdum,
pretium purus vitae, fermentum risus. Sed facilisis vehicula arcu vitae
efficitur. Donec aliquet auctor dui in blandit.

\par\rule{\textwidth}{0.5pt}


\vspace{1cm}

\begin{flushright}
 % by-author
\end{flushright}

\end{titlepage}\renewcommand*\contentsname{Table des matières}
{
\hypersetup{linkcolor=}
\setcounter{tocdepth}{0}
\tableofcontents
}
\begin{table}

\caption{\label{tbl-table1}}

\centering{

\captionsetup{labelsep=none}

\begin{longtable*}{lrl}
\toprule
a & b & c \\ 
\midrule\addlinespace[2.5pt]
text & 1 & Long text aaaaaaaaaaaaaaaaaaaaaaaaaaaaaaaaaaaaargh \\ 
others & 2 & Long text aaaaaaaaaaaaaaaaaaaaaaaaaaaaaaaaaaaaargh \\ 
\bottomrule
\end{longtable*}

}

\end{table}%

texte avant

\begin{tcolorbox}[enhanced jigsaw, arc=.35mm, bottomtitle=1mm, rightrule=.15mm, coltitle=black, title={titre de l'encadré}, colbacktitle=quarto-callout-tip-color!10!white, breakable, colframe=quarto-callout-tip-color-frame, leftrule=.75mm, toprule=.15mm, bottomrule=.15mm, opacityback=0, left=2mm, opacitybacktitle=0.6, toptitle=1mm, titlerule=0mm, colback=white]

\begin{Shaded}
\begin{Highlighting}[]
\FunctionTok{library}\NormalTok{(tidyverse)}
\FunctionTok{ggplot}\NormalTok{(cars)}\SpecialCharTok{+}\FunctionTok{geom\_point}\NormalTok{(}\FunctionTok{aes}\NormalTok{(}\AttributeTok{x=}\NormalTok{speed, }\AttributeTok{y=}\NormalTok{dist))}\SpecialCharTok{+}\NormalTok{ofce}\SpecialCharTok{::}\FunctionTok{theme\_ofce}\NormalTok{()}
\end{Highlighting}
\end{Shaded}

\includegraphics{test_files/figure-pdf/unnamed-chunk-2-1.png}

blablbl

\begin{table}[H]

\caption{\label{tbl-table2}}

\centering{

\captionsetup{labelsep=none}

\begin{longtable*}{lrl}
\toprule
a & b & c \\ 
\midrule\addlinespace[2.5pt]
text & 1 & Long text aaaaaaaaaaaaaaaaaaaaaaaaaaaaaaaaaaaaargh \\ 
others & 2 & Long text aaaaaaaaaaaaaaaaaaaaaaaaaaaaaaaaaaaaargh \\ 
\bottomrule
\end{longtable*}

}

\end{table}%

\end{tcolorbox}

continuing



\end{document}
